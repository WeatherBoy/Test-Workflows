\documentclass{article}
\usepackage[english]{babel}
\usepackage[utf8]{inputenc}
\usepackage[top=2cm, bottom=3cm, left=2cm, right=2cm]{geometry}

% ## Inserting Code ##
\usepackage{pythonhighlight} % <-- for inserting Python code

% ## Nice referencing ##
\usepackage{csquotes}   % <-- for biblatex
\usepackage{biblatex}
\addbibresource{Bibliography.bib}

% ## Math ##
\usepackage{amssymb}
\usepackage{amsmath}
\usepackage{amsthm}
\usepackage{cancel} % <-- the big slash over equations
\usepackage{breqn}  % <-- Automatic wrapping of long equations (good stuff)

% ## For the graphical aspect ##
\usepackage{float}  % <-- superior package for figure handling
\usepackage[dvipsnames]{xcolor}
\usepackage{graphicx} % <-- Required for inserting images
\usepackage{wrapfig} % <-- In case you want to wrap text around figures (can look quite pro)
\usepackage{tabularx} % <-- Better tables (supposedly)
\usepackage{booktabs} % <-- For better looking tables
\usepackage{subcaption} % <-- For subfigures
\usepackage[many]{tcolorbox} % <-- For making boxes around text

% ## Referencing ##
\usepackage{fancyref}   % <-- Supposed to make referencing better, don't know where it's used
\usepackage{hyperref}   % <-- has \autoref{} and all kinds of other goodies
\usepackage{url}   % <-- web addresses are handled properly when URL referencing

% ## Utility ##
\usepackage{enumitem}   % <-- enumerate, itemize and description
\usepackage{pdfpages}   % <-- for including external PDFs
\usepackage{titlesec}   % <-- (used for the header) section and chapter titles manipulation
\usepackage{subfiles}   % <-- smart for BIG projects where you include multiple subfiles
\usepackage{comment}    % <-- multi-line comments
\usepackage{fontawesome5}   % <-- add cool logos to your report
\usepackage{sectsty}    % <-- can help with generalising section design

\usepackage{fancyhdr}   % <-- for the header and footer
\pagestyle{fancy}   % <-- used to great extent for the title page header
\usepackage{minted} % <-- for inserting code with syntax highlighting
\setminted{breaklines, fontsize=\small}

% Make github logo
\usepackage{fontawesome5}
\newcommand{\github}[1]{%
   \href{#1}{\faGithubSquare}%
}

% For math definitions
\newtheorem{definition}{Definition}[section]
\newtheorem{theorem}{Theorem}[section]
\newtheorem{corollary}{Corollary}[section]
\tcolorboxenvironment{definition}{
  sharp corners,
  boxrule=0.4pt,
  colback=white,
  before skip=\topsep,
  after skip=\topsep,
}
\tcolorboxenvironment{theorem}{
  sharp corners,
  boxrule=0.4pt,
  colback=white,
  before skip=\topsep,
  after skip=\topsep,
}
\tcolorboxenvironment{corollary}{
  sharp corners,
  boxrule=0.4pt,
  colback=white,
  before skip=\topsep,
  after skip=\topsep,
}


% ## For pseudocode ##
\usepackage{listings}
\lstset{
  basicstyle=\small\ttfamily,
  numbers=left,
  numberstyle=\tiny,
  stepnumber=1,
  frame=single,
  columns=flexible,
  keywordstyle=\bfseries\color{blue!70!black},
  commentstyle=\itshape\color{green!40!black},
  language=Python,
  showstringspaces=false
}


% Formalia goes here:
\newcommand\datoen{September 1, 2025} %<--- Dato
\newcommand\opgavetitel{Standardising Patient Data in Danish Healthcare: A FHIR-Based Prototype for GP Reports} %<--- Navnet på opgaven


% Make GREAT looking header and footer:
\fancyhead[L]{Project Course}
\fancyhead[R]{\rightmark}

\begin{document}
    \begin{titlepage}
    \raggedright
    \includegraphics[width=0.15 \textwidth]{figures/Titlepage/Coral_Red_RGB.pdf}
    \hfill
    \includegraphics[width=0.45\textwidth]{figures/Titlepage/tex_dtu_compute_a_uk.pdf}
    \begin{center}
        \vspace{3 cm}
        \hrule
        \vspace{.3cm}
        { \huge {\bfseries {\opgavetitel}}
        } 
        % Title of the report:
        %The impact of regularization on classifiers and explaining the explainability of axiomatic attributions
        
        \vspace{.1cm}
        { \LARGE {\bfseries 
            {
                Human-Centered Artificial Intelligence
            }
        }
        }
        \vspace{.5cm}
        
        \hrule
        \vspace{1.5cm}
        
        \textbf{Authors}\\
        \vspace{.5cm}
        \centering
        
        % add your name here
        Felix Bo Caspersen | s183319\\
        
        \vspace{1.5cm}
        
        \textbf{Supervised by}\\
        \vspace{.5cm}
        \centering
        
        Professor Per Bækgaard \\
        \vspace{1.5cm}
        
        Kongens Lyngby \\
        \centering \datoen % Dags dato
    \end{center}
\end{titlepage}
    \begin{abstract} 
    This project explores the potential of data standardisation and user-centred design in the Danish healthcare sector. The work builds on prior thesis research and focuses on the dual challenges of inaccessible clinical data and the lack of interoperable digital tools for general practitioners (GPs). In the absence of direct access to clinical datasets, the study applies a case-based approach centred on diabetes care. Eight validated questionnaire instruments from the DiaFocus project were implemented in a Python-based pipeline, producing automatically generated HTML reports. These reports were tested in qualitative user interviews with three medical students to evaluate their adequacy for supporting GP consultations. The results suggest that a standardised, automatically generated report can provide value in summarising patient data, though the findings remain exploratory due to the limited user pool. To complement this prototype, a FHIR-based data model was developed to demonstrate how the required data could be retrieved in a standardised manner. The project concludes that while broader adoption requires clinical validation and collaboration with healthcare institutions, the approach offers a promising foundation for improving data accessibility, interoperability, and the user experience of digital tools in Danish general practice.
\end{abstract}
    \section*{Preface and Acknowledgements}
This project course is written as part of the Master of Science in Human-Centered Artificial Intelligence at the Technical University of Denmark, Department of Applied Mathematics and Computer Science. Professor Per Bækgaard supervised this project. The project was completed from the 1st of April 2024 to the 1st of September 2025 and amounts to 10 ECTS credits. All the code supporting the project is in the GitHub repository \github{https://github.com/WeatherBoy/Test-Workflows}\footnote{For completeness: \href{https://github.com/WeatherBoy/Test-Workflows}{https://github.com/WeatherBoy/Test-Workflows}.}.
\\
\\
Firstly, I would like to thank the three medical students who took the time to provide counsel and for the interviews, which provided invaluable feedback for this report. Secondly, I would like to sincerely thank my supervisor, Per Bækgaard. I believe his calm attitude, vast knowledge on the subject, and boundless patience to have been the perfect match for my chaotic approach to this last project of my university degree.

    \setcounter{page}{1}
    \pagenumbering{roman}
    \tableofcontents
    \newpage

    \setcounter{page}{1}
    \pagenumbering{arabic}
    \section{Introduction}
This project is, in large part, a continuation of my Master's thesis ~\cite{thesis}, which concerned the development of digital applications for the healthcare sector. Throughout the process of writing my thesis, I became increasingly aware that the bottlenecks wouldn't be in developing the software. Instead, the main impediment was twofold; one was bureaucratic in nature - a human factor - while the second was technical in nature - the accessibility, or perhaps more so the inaccessibility, of data.
\\
The bureaucratic aspect consisted of miscommunication concerning application functionality between myself and stakeholders from the healthcare sector, in addition to misalignment between individual goals amongst personnel and broader goals at the administrative level from the industry side. Yet, as I have touched on this subject plentily in my thesis, I will refrain from diving further into it in this project. 
\\
\\
On the other hand, the technical difficulties, namely, data inaccessibility, are no less relevant. Therefore, following my thesis, I have once more reached out to the industry - the healthcare sector - in an attempt to assist in bridging the gap between developers and healthcare data. Seeing as data inaccessibility is not only an interfacing issue, but, maybe more prevalently, also a concern of standardisation, expert feedback, or collaboration, would contribute enormously to the endeavour of data accessibility.
\\
However, it proved extremely difficult to establish an industry contact. Therefore, this project primarily concerns itself with user experience (UX) before, during, and after a patient-medic checkup. Here I have analysed the UX aspects of data broadcast through infographics, and whether the user found the infographic adequate regarding the medical checkup. Furthermore, I have explored an international data standard, dubbed the Fast Healthcare Interoperability Resources (FHIR), and how the data used in the aforementioned infographics could be retrieved using FHIR.
Due to the nature of the project, the user in question, for these analyses, has been the healthcare personnel. And as I was unable to collaborate with the industry, I have instead been very fortunate in getting to interview and cooperate with a handful of my friends who are all studying to become medical practitioners.   

    \section{Theory}\label{sec:theory}
\subsection{Fast Healthcare Interoperability Resources (FHIR)}
Fast Healthcare Interoperability Resources (FHIR) is a standard for healthcare data, designed with the intent of enabling secure digital exchange.
\\
Unlike most other products and industries, \emph{standards} thrive on monopoly. Monoculturalism is a widely debated topic and probably most commonly agreed upon as unhealthy or counterproductive in the grand scheme of things. In that sense, we, as humans and a collective, benefit from the development of multiple languages and cultures. However, when it comes to standards, especially something as critical as healthcare data, we all benefit from speaking exactly one language\footnote{One might be tempted to consult the XKCD comic of appendix \autoref{sec:XKCD-standards}.}. A singular language, one standard, enables seamless transfer of information between different institutions and even across country borders. This, in turn, permits third-party developers to more readily design software applications harnessing the statistical advantages associated with large quantities of standardised data. Furthermore, if everybody utilised one standard, these technological advancements could also propagate across institutions and national borders. Taking into account this strange tolerance of monopolies, in order to understand why FHIR is \textbf{the} standard, one must necessarily also get acquainted with the organisation developing it, Health Level Seven (HL7).
\\
FHIR was created and is currently maintained by the American non-profit organisation, HL7. It was founded in $1987$ by Donald W. Simborg with the original objective of developing a standard for the exchange of data within Hospital Information Systems\footnote{HIS is a somewhat broad term, describing the elements of health informatics which mainly focus on the administrative needs of hospitals. A HIS is a comprehensive integrated system, which has the intended function of managing all operational aspects of a hospital, i.e. medical, administrative, financial, legal and issues pertaining to the processing of services. HIS are also commonly referred to as Hospital Management Software or a Hospital Management System~\cite{Wiki-HIS}.} (HIS)~\cite{Wiki-HL7}. As of $2025$, today HL7's vision, as stated on their official website~\cite{HL7-about}, is:
\begin{center}
    \emph{"A world where every individual has secure, timely, and appropriate access to accurate health information, empowering them, their care teams, and communities to make informed decisions anytime, anywhere."}
\end{center}
As a testament to their vision, in September $2012$ HL7 announced their intention to license their standards at no cost, and finally in April $2013$ they made all their primary standards, alongside other select products, of which FHIR is a part, licensable at no cost~\cite{HL7-no-cost}. Furthermore, as an American-founded and American-based organisation, HL7 is accredited by the American National Standards Body (NSB) of the International Organisation for Standardisation (ISO), namely the American National Standards Institute (ANSI). 	
    \section{Methodology}
\subsection{Reaching out to the industry}
While writing my thesis, I hoped that the software solutions I was developing could build the foundation for a start-up. However, multiple complications arose. I ended up working on three different projects, and even the "easiest", at least from a software development point of view, quickly hit a wall. It ultimately proved too difficult to implement, simply because the data sources were too inaccessible. While the two other projects might have worked in theory, they would still be riddled with infant diseases as soon as they reached production, due to the data inaccessibility issues. Seeing as the healthcare sector is already full of disjoint software solutions, usually designed with clunky or, at best, simply odd workarounds, I elected to refrain from contributing my own disjoint software solution and thus abandoning the start-up dream.
\\
Yet, it remained clear to me that this underlying issue of data inaccessibility persisted. Therefore, I became quite hopeful upon reading the 2024 report by "Danske Regioner" ~\cite{Den-Reg-digitalisation}. In the report, "Danske Regioner" proposed a so-called "Danish Healthcare Cloud". I have since come to think of the \textbf{Healthcare Cloud} as a pseudo-appstore; however, it might be more aptly described as a combination of a data warehouse, or data lake, and an appstore. The core functionality of the \textbf{Healthcare Cloud} would be a platform which allowed for easy data sharing. In other words, all applications on the \textbf{Healthcare Cloud} would be required to store their data in the \textbf{Healthcare Cloud}, such that any other application on the \textbf{Healthcare Cloud} would have the same data readily accessible. This would immediately eliminate any need for the healthcare personnel to perform time-wasting double (triple or even more) loggings of the data. However, the \textbf{Healthcare Cloud} would also need to support a universal login for the healthcare personnel, such that they don't have to keep separate log-ins for each platform. Additionally, the \textbf{Healthcare Cloud} would have to handle some kind of licensing to verify that it is the right developers who can create platforms with access to the data.
\\
Upon finishing my thesis in March of 2025, I sought to initiate contact with "Danske Regioner" in the hopes that I might be able to contribute to the project. Unfortunately, "Danske Regioner" isn't currently working on that project. Instead, they had simply collaborated with a tech consultancy to outline the \textbf{Healthcare Cloud}, what problems it could solve and how it could be developed. "Danske Regioner" responded kindly to my message and advised that I should reach out to CIMT ("Center for IT \& Medico-Teknologi") in the capital region. They deemed that CIMT might be able to allocate resources for such a project.
\\
It took roughly two months before I got to have a meeting with CIMT. They were two very nice and formal representatives who, in some capacity, worked with planning and scheduling data within the healthcare sector of the capital region. However, although they also found the project of interest, they very quickly shut it down. They informed that CIMT had neither the resources nor the authority to take on a project such as the \textbf{Healthcare Cloud}.
\\
Another big aspect of realising the \textbf{Healthcare Cloud} is data standardisation. Unless the healthcare institutions can somehow come to an agreement on how to label all data points, the \textbf{Healthcare Cloud} would still not allow for cross-institutional data sharing. So even if CIMT managed to standardise the healthcare data within the capital region, it would remain incompatible with the data from the other four regions. As such, CIMT ultimately rerouted me back to "Danske Regioner", as, if anybody had to "make the decree", so to speak, of data standardisation, it would be them.
\\
Upon reaching out to "Danske Regioner" once more, I was, to my great joy, met by a very kind secretary, who remembered my original call. She seemed impressed that I had managed to secure a meeting with CIMT, and she promptly directed me further up the administrative chain. Luckily, this time it was simply a phone call, and it didn't take two months to plan. However, the administrative worker I got a hold of at "Danske Regioner" stated that, although it was the staff at CIMT's impression, "Danske Regioner" doesn't handle administrative software on that kind of scale. Instead, I was redirected to the Danish Health Data Authority (DHDA).
\\
I was fully aware that DHDA wasn't working on a \textbf{Healthcare Cloud}, so when I reached out to them, it was more so in the hopes of creating a short collaboration for a 10 ECTS course at DTU. Maybe they could provide some actual or mock data from the Danish healthcare sector, and hopefully, I could contribute something useful with that. Unfortunately, that wasn't the case. DHDA found my motivation endearing and recommended that I apply for a job at their department, as they hoped there would come an opening where I would have the possibility of working on something adjacent to the \textbf{Healthcare Cloud}. Nevertheless, DHDA proved a dead end all the same, and as such, I was unable to secure data for this project from an official source. 


    \section{Analysis}
In the analysis, I will mostly cover the interviews I performed and the resulting insights. Firstly, I'll go over the answers I got from the single individual, with whom I went through all eight questionnaires. 

\subsection{The questionnaire responses}
The answers themselves aren't all that interesting on their own. Without "de-anonymising" the participant too greatly, I can briefly add that they have lived with type 1 diabetes for over thirty years, and as such, they have built strong habits. Therefore, their day-to-day isn't vastly negatively affected by the disease. Also, as alluded to by the comment to the B2-6 question of the B2 - Emotional Distress questionnaire\footnote{Which can be seen in appendix \autoref{sec:B2-emotional-distress-answers}.}, they are instead grateful that it can be "fixed" to such a strong degree and that they still get to live their life the way they want.
\\
However, after roughly an hour of interview and going through every question, I was definitely left with greater insight. First and foremost, answering all the questions in one sitting, in a row, was definitely never the intended way. As it is also stated in the DiaFocus report, the D1 (Food Behaviour), D2 (Pittsburgh Sleep Quality Index (PSQI) ), D3 (Hospital Anxiety and Depression Scale (HADS) ) and D4 questionnaires were issued based on what the patient themselves found to be important areas of concern (as answered by the C - Areas of Concern questionnaire). Additionally, the B1 - Life Style Information questionnaire would only be answered initially. After an initial consultation with a GP, the patient was then expected to answer the B2 - Emotional Distress and B3 - Well-Being (WHO-5) continuously, as in, on a weekly, bi-weekly or monthly basis, alongside the agreed-upon, between patient and GP, selection of D1, D2, D3 and D4.
\\
In \autoref{fig:diafocus-questionnaire-flow} can be seen the intended flow of the questionnaires as per the DiaFocus project~\cite{DiaFocus}.

\begin{figure}[H]
    \centering
    \includegraphics[width=0.75\textwidth]{figures/analysis/diafocus_questionnaire_flow.png}
    \caption{DiaFocus questionnaire flow.}
    \medskip
    \small
    \raggedright
    Description from the DiaFocus report: "The flow of questionnaires in DiaFocus starts from when the user installs the app, signs in, and starts using the application.". The image is taken from ~\cite{DiaFocus}.
    \label{fig:diafocus-questionnaire-flow}
\end{figure}

\noindent
As for the B4 - Diet \& Exercise and B5 - Blood Glucose, these were intended to be sampled in cooperation with the patient's mobile OS for the DiaFocus application, or in the case of B5, manually filled in, if the patient doesn't have an insulin pump. This is also why both of these quite important data points are lacking from my final HTML report.
\\
\\
For full transparency, I altered the setup a bit for the HADS and WHO-5 questionnaires post the interview. Originally, I had made a carbon copy of the DiaFocus' instruments; however, according to the WHO documentation~\cite{WHO5}, a higher score indicates a better well-being. I then changed both the instrument of appendix \autoref{sec:who-5} and the answers of appendix \autoref{sec:who-5-answers} accordingly. Albeit, I completely understand the motive behind reversing the scores, as they would then be directly comparable with the results of the other questionnaires. I "solved" this by reversing the score of the WHO-5 for the HTML report. 
\\
For the HADS, the objective truth seems murky. Quite a lot of sources~\cite{Wiki-HADS, DiaFocus, HADS, HADS-StrokeEngine, HADS-2017, HADS-Danish} agree on there being a verified instrument called the Hospital Anxiety and Depression Scale, which has 14 questions, rated on a scale of $[0, 3]$, with questions split evenly between the areas of anxiety and depression. From the original work, Zigmond et al.~\cite{HADS}, one might infer that a higher numeric score indicates greater signs of anxiety and depression, based on their indicated cut-off points and the phrasing of the numeric value to phrase translation in the appendix. Regardless, this is never stated explicitly.  
\\
Additionally, from Zigmond et al.~\cite{HADS}, the instrument wasn't made with a single uniform translation for every question. Instead, almost every question had a unique numeric value to phrase translation. Combined with multiple sources~\cite{HADS-StrokeEngine, HADS-Danish}  stating that some of the questions were reverse scored, yet differing in the exact amount, this leaves a murky image.
\\
I elected to reverse the overall scoring of the HADS, post responses and even change the wording, such that it was more in line with a majority of sources~\cite{HADS, HADS-StrokeEngine, HADS-2017, HADS-Danish}. From my own intuition, and believing a handful would be flipped in score, I elected to choose that question four\footnote{"I can sit at ease and feel relaxed."} of the anxiety instrument, was flipped and that questions one\footnote{"I still enjoy the things I used to enjoy."}, 
two\footnote{"I can laugh and see the funny side of things."}, three\footnote{"I feel cheerful."}, and six\footnote{"I look forward with enjoyment to things."} and seven\footnote{"I can enjoy a good book or radio or TV program."} was flipped.
\\
This is how it has been scored for the final HTML report moving forward.
\\
\\
Finally, I would like to address why I have both the B2 and PAID instruments, as well as answers for both. In short, I was too quick when going through the DiaFocus report initially. I hadn't noticed that the B2 instrument presented in Appendix B~\cite{DiaFocus} was an exact copy of the PAID-6 presented in the clinical report of Appendix D~\cite{DiaFocus}. This mishap led me to ask the exact same questions twice during the interview. However, we went through the questionnaires in the same order as presented in the Appendix \autoref{sec:questionnaire-instruments}. Therefore, we went through the "B2 - Emotional Distress" very early, and after roughly 40 minutes, we went through the "Problem Areas in Diabetes". Both the participant and I noticed that the questions were very similar, and both expressed that we felt we had been over them before. However, I believe that, coupled with the fact that it was the last questionnaire, we went through them anyway.
\\
Now, why keep both? Why not just toss one and not mention it? I might have. Were it not for the fact that they are distinct. Despite having the exact same questions and the exact same answer scale, the participant was slightly more negatively inclined on the second go than they had been 40 minutes earlier. More precisely, for B2, all but the second question was rated $0$ out of $[0-4]$. The second question was rated a $1$. For PAID, only the fourth and fifth questions were rated as $0$, the remaining were now at a $1$.
\\
This is quite pseudo-scientific, but I mention this as one of the medical students I interviewed also noted that people tend to answer more negatively when continuously probed on matters which are usually associated with negative emotion.

\subsection{Testing the HTML report}
I interviewed a total of three medical students. I did this by, firstly, explaining to them the context in which it was intended to be used. They all had some idea, as I have consulted all of them previously on this project; however, they didn't know that the HTML report was designed with the intent of assisting a GP in consultation with a diabetes patient. Then, secondly, I asked them to, to the best of their abilities, think out loud when going through the report. I would like to note that I tried to give them as little help as possible in navigating the HTML report, as, after all, the intent was to assess the UX. Yet, I'm fully aware that as a close friend of each of these three medical students and as the designer of the HTML report, I am in no way, shape or form an unbiased interviewer. 
\\
After an initial go through, I went a bit more in-depth and asked them, "What did you like?", "What, if anything, did you think was missing?", "Would you use this if you were a GP who had a consultation with a diabetes patient?".
\\
In the following, I will firstly cover their individual initial thoughts, then I will go through an overall assessment and cover the comments they had in common. Furthermore, I will refer to them as "A", "B" and "C" alphabetically ordered by the order in which they were interviewed.

\subsubsection*{Interview with "A"}
When I interviewed "A", the HTML report was not yet in the state that can be seen in Appendix \autoref{sec:html-report-pdf}. Then it was an almost solely ChatGPT-generated prototype with hallucinated questions and responses. However, it still contained the primary questionnaire structure and the bar plot, which can be seen at the top of the HTML report. Then I tried to largely replicate the DiaFocus clinical report header, of which I believe the inspiration is still noticeable in the final HTML report. "A" noted that it seemed strange to include the phone number and email address for the patient. Consequently, I made an A and B version of the HTML report. The A-version has a very information-heavy header, and the B-version, which is the one in the Appendix, has only the most crucial information present. "A" noted that the "Next consultation" information element of the header seemed smart. As if it were "N/A" or maybe "Not planned", then it would be a good reminder that planning the next consultation would be a key aspect of the current consultation. Admittedly, although it might be a cool feature, I manually included it through a metadata.json file placed in the same directory as the response JSON (the one named "YYYY-MM-DD.json"). The structure might be the same for a production-ready system, but automatic updates would likely require an integration with the healthcare institution where the consultation was performed. Based on the experience from my thesis, this prospect seemed like a wild pipe dream. "A" also noted that they liked the feature where, if you click on a specific bar of the bar plot, you are linked down to the actual questionnaire responses related to the specific score indicated by the bar plot. Finally, they added that they figured a yearly update on the "B1 - Life Style" might be productive as it could inform the GP whether they were still smoking, had stopped or might have begun. However, we also discussed the possibility of dynamically tweaking the B1 questionnaire, such that questions like exercise were exempted if they had continuously provided fitness data to the underlying system through a cooperation with their mobile OS (the questionnaire the DiaFocus report refers to as "B4 - Diet \& Exercise").  

\subsubsection*{Interview with "B"}
When I interviewed "B", the HTML report was almost identical to that of the appendix. The only difference is that I, at the time, had a tough bug with regard to extracting the information correctly from the JSON files. Therefore, I had hardcoded 1-2 questions per questionnaire directly into Python. It was neither good code nor a complete HTML report, but it got the job done as a prototype. "B" was quite thorough on the subjects they found missing and elaborated in greater detail on the subjects which "A" and "C" might only have mentioned in parsing. In turn, they didn't have a plethora of additional comments. "B" noted that if they wanted to go over the questionnaire responses with the patient, then the HTML report gave a very nice overview. They also noted that if they, as a GP, were concerned as to their patient's health and probably would have asked the questions at a consultation regardless, then it seemed like a nice timesaver to have them asked beforehand. Finally, "B" noted that in the case that the patient doesn't have an insulin pump, then it would be valuable information to have that noted in the HTML report, as it heavily pertains to how well they are at regulating their own insulin levels.

\subsubsection*{Interview with "C"}
When I interviewed "C", the HTML report was identical to that of the appendix. With regards to the design "C" noted a lot of the same facets as "A" and "B" had pointed out. Yet, they also stated that they found it unrealistic that the average patient would answer so many questions in between consultations. However, if the data were to be present, then they figured it was a cute setup. As the only one, "C" mentioned that they would like the underlying system to catch or flag possible connections. E.g. they pointed out that if the patient had been reporting an increasingly worse sleep score but also gained a lot of weight and exercised less within the same period, then there could be a correlation. Instead of "C" themselves spending time on analysing and finding this correlation, they would like for the system to be able to selectively pinpoint some problems like this (if present, of course). This, however, would require the HTML report to support a temporal data setup, which it currently doesn't and which I will elaborate further on in a moment. Finally, "C" noted that, like a system for detecting correlations, they would like for certain questions to be flagged as 'critical'. In particular, they mentioned the "Cannot breathe comfortably" question of the PSQI, which she would find very alarming if scored poorly, even if the overall PSQI score might not be within a critical threshold. We, in turn, also both agreed that defining such critical questions would require a more in-depth analysis and clinical expertise.

\subsubsection*{General feedback - temporal data, examination results, other chronic illnesses \& polypharmacy}
All three of the medical students were in agreement that the current v.2b of the HTML report, as can be seen in Appendix \autoref{sec:html-report-pdf}, definitely couldn't serve as an alternative to "Sundhedsplatformen" (which I can't but agree with as well). "C" explicitly stated that this might be a tool they would consider using if they knew the patient had had troubles with their mental health, sleep and adhering to both prescribed medication and a diet. Yet, even still, "C" would only look at it if they had spare time, which they didn't find to be expected within the healthcare sector. However, all three of them also independently agreed that if temporal data, medical examination results and a list of both medications and possible other chronic illnesses were to be included, then this would prove a far more valuable tool.
\\
Temporal data would be in the form of 1) the ongoing results from the questionnaires, 2) the data gotten from the OS, e.g. the B4 questionnaire from the "Continuous Assessment" section of \autoref{fig:diafocus-questionnaire-flow}, and 3) data gotten from external applications, such as the blood glucose measurements and perhaps data from a sleep app. This way, it would be possible to see a development, rather than the current setup, which solely provides a snapshot.
\\
Examination results would be like journal entries, but for the necessary and crucial examinations regarding diabetes patients. These are the eye, heart, kidney and feet examinations, as noted by the Official Danish Medical Handbook (Lægehåndbogen)~\cite{SundhedDK-diabetes}, performed in order to spot any issues which commonly lead to long-term medical complications for diabetes patients. "B" noted that he assumed that "B" themselves would perform the eye and foot examinations during consultations, but regardless of whether prior examinations were present, "B" would like to have them included in the HTML report. 
\\
A list of other chronic illnesses and prescribed medications is probably quite self-explanatory; however, they would help in assisting in how to treat the diabetes patient. "C" noted that it would be especially nice if the system itself provided a feature whereby one could type in a medication and the system could give a warning if there was a clash with any other previously prescribed medications. Both "A", "B", and "C" also noted that a list of other chronic illnesses could help give the GP a better overview and maybe find a possible causality for any sudden developments in the temporal data.

\subsection{Putting it all together - A FHIR-based datamodel for the HTML report}

    \section{Discussion}
Throughout this report, I've tried my hand at many different topics, ranging from researching and initialising a FHIR-compatible data warehouse-esque prototype, to designing and implementing a standardised HTML report for clinical use, to conducting user experience (UX) interviews with medical students. These varied endeavours have led to a lot of insights in terms of technical feasibility, domain knowledge, as well as user experience. In this section, I will address some of the more questionable choices made throughout, in addition to discussing prospects for future work.

\subsection{Main discussion points}
\subsubsection*{Biased user feedback}
As previously stated, for this report, I was unable to acquire a source of clinical data and/or expertise. Therefore, I ultimately relied on being able to interview a handful of medical students. While these students were all very knowledgeable and forthcoming with their feedback, they were, after all, still students. As such, it would be na\"ive to assume that their clinical experience, and by extension, their feedback, is on par with that of seasoned medical practitioners. Furthermore, as I am close friends with each of these students and know them intimately, there is the obvious risk that they might have been inclined to give more positive feedback than they would have otherwise.
\\
Although the setting for testing and data collection was not ideal, I am certain that the provided feedback was nevertheless valuable. The domain knowledge gathered from the interviews showcased numerous points of improvement for the HTML report, in addition to validating the overall concept of a standardised report. However, for further development of the concept, testing with actual medical practitioners would be a top priority.

\subsubsection*{Choice of coding setup}
With regards to the choice of setting up the standardised information, or infographic, as an automatically generated HTML report, based on the questionnaire data in a JSON format, an obvious question might be: Why?
\\
Initially, I attempted to play around with Figma, as I have had some prior experience from a course at DTU. Yet, after roughly an afternoon's worth of tinkering, I was simply left frustrated, as I felt the base version of Figma didn't provide, or at least not in an intuitive manner, the functionality I required. I also considered using Microsoft's PowerPoint for creating a mockup of the infographic, as I have quite substantial experience with this software. Again, I toyed around with it for a handful of hours; however, being a programmer at heart, I was irked by the lack of automation. I had already spent a considerable amount of time writing the questionnaire responses down and couldn't be bothered to format them nicely within PowerPoint. Instead, I went for a more technical approach by compiling the data into a JSON format and writing a script to generate the HTML report automatically.
\\
Despite this by no means being the most seamless approach for intricate UX testing, I stand by my choice. The codebase serves as a solid foundation for further development, and while building it, I developed a far better relationship with HTML code through using Python's \texttt{Jinja2} templating engine. Whereas I had previously always looked at HTML code as a jumbled mess of gibberish, I now feel I have a novice understanding of how it works. Not to mention, the choice of HTML as the output format opens up a plethora of possibilities for further development, seeing as HTML is a very flexible format that can be rendered on virtually any device with a screen and a web browser.

\subsection{Future work}
\subsubsection*{Data retrieval from external applications}
In the DiaFocus project~\cite{DiaFocus}, they mention how uncollaborative they found the users' (diabetes patients) mobile OS to be. Although the report is from 2023, I doubt that much has changed since then. This brings about the question of how much data can be retrieved not only from the mobile OS itself, but rather external applications. Maybe the user has a fitness app that tracks their steps, heart rate, and sleep patterns. If this data could be retrieved and integrated into the report, it would provide a more comprehensive view of the patient's health and lifestyle. However, firstly it would be prudent to assess through research how much additional value this data would provide, as it would be a significant undertaking to implement such functionality. There would be privacy concerns to consider, as well as the technical challenge of integrating with various third-party applications. In turn, this would very much not be in the spirit of standardisation, seeing as there are countless fitness and health tracking apps available, each with their own data formats and APIs. However, if a few popular apps could be integrated, it might be worth the investment.

\subsubsection*{Summarise medical journals}
Whilst still trying to figure out how to best work on some of the standardisation and UX testing within the Danish healthcare sector, I sat down with one of the medical students I would later interview regarding the HTML report. They then had a laptop with access to "Sundhedsplatformen" (SP), with which the student and I went through some of the core functionalities and setup. They showed me how they would typically use SP to look up patient information, such as previous diagnoses, prescriptions, and lab results.
\\
There has been a great deal of commotion surrounding SP, and I have already been over this in my thesis; hence, I will simply discuss a feature that might be integrated atop SP, rather than a fundamental change. First and foremost, for legal purposes, I obviously have to declare that I was only presented mock data. Secondly, I have to admit, everything I had heard about SP made it seem way worse than when I got to look around on the platform myself. There was quite a lot of functionality. However, it seemed, from my friend, that the greatest issue presented by SP was that not everyone uses it. It is very common for hospitals in the Capital Region to use SP, but GPs might use their own tools, which means data isn't necessarily transferable from individual clinics to hospitals. In addition, there are simply a lot of regions which doesn't use SP as they have their own system and they consider it superior. The biggest potential we discussed, regarding SP, was to find a way to somehow summarise patient journals. The workflow for medical practitioners at the moment seems to be, preferably, to completely ignore SP and simply talk to the colleague who has had the latest contact with your patient, but if that isn't possible, then look at the latest journal entry, and if there have been a lot of entries, then pluck out a couple for deeper inspection. For this, we briefly discussed the proposition of some kind of Large Language Model (LLM) in combination with a stricter journaling standardisation, e.g., by the use of FHIR to create a summary of journal entries for the medical practitioner. 
\\
With neither a comprehensive theoretical preface nor a thorough analysis, I simply had a gut feeling that if the journals were to be standardised, say using FHIR, then they could be processed through a combination of an LLM and some rule-based system to create a summary of the most important points for the medical practitioner. Although it isn't directly related to creating a standardised report for clinical use, I find this an incredibly interesting avenue for future work. It would require a great deal of development, standardisation, testing, and, necessarily, collaboration with the healthcare sector, but I believe it has the potential to significantly improve on a very tedious and time-consuming task for medical practitioners
    \section{Conclusion}
In this report I have addressed the issue concerning data standardisation and accessibility within the Danish healthcare sector. In particular, in an attempt to alleviate some of the everyday tasks for healthcare personnel, I set out to answer the following questions:
\begin{itemize}
    \item Is it possible to define a common care pathway for multiple patients that broadly requires the same data for their GPs?
    \item Can such data be presented in a standardised HTML report that is adequate for clinical use?
    \item Can the data required for the HTML report be retrieved using FHIR?
\end{itemize} 
To this end, I have researched the FHIR standard and its applicability for retrieving patient data, in addition to designing and implementing a prototype HTML report for clinical use and conducting UX interviews with medical students.
\\
Regarding the first question, although I didn't find a common care pathway as I had initially expected, I did find that both diabetes patients and pregnant women serve as good examples of patient groups that broadly require the same data for their GPs. In particular, I tackled the diabetes use case, and hope that the findings can provide a foundation for further work on standardised reports for other patient groups.
\\
With respect to the second question, I wrote a codebase to automatically generate an HTML report based on questionnaire data in a JSON format. Although I believe this to be a proof of concept for a standardised HTML report, I lacked the necessary clinical resources to properly validate the report for clinical use. However, through UX interviews with medical students, I was able to gather valuable feedback and insights that can be used for further development of the concept.
\\
Finally, regarding the third question, I researched the FHIR standard and found that it is indeed possible to retrieve the necessary data for the HTML report using FHIR resources as exemplified by the datamodel diagram in \autoref{fig:datamodel-diagram} of the analysis section.


    \listoffigures

    \listoftables

    \section{Software versions used}
\begin{itemize}
    \item \texttt{plotly}: v6.3.0
    \item \texttt{jinja2}: v3.1.2
    \item \texttt{playwright}: v1.55.0
\end{itemize}
    \section{Appendix}


    \printbibliography
\end{document}