\begin{abstract} 
    This project explores the potential of data standardisation and user-centred design in the Danish healthcare sector. The work builds on prior thesis research and focuses on the dual challenges of inaccessible clinical data and the lack of interoperable digital tools for general practitioners (GPs). In the absence of direct access to clinical datasets, the study applies a case-based approach centred on diabetes care. Eight validated questionnaire instruments from the DiaFocus project were implemented in a Python-based pipeline, producing automatically generated HTML reports. These reports were tested in qualitative user interviews with three medical students to evaluate their adequacy for supporting GP consultations. The results suggest that a standardised, automatically generated report can provide value in summarising patient data, though the findings remain exploratory due to the limited user pool. To complement this prototype, a FHIR-based data model was developed to demonstrate how the required data could be retrieved in a standardised manner. The project concludes that while broader adoption requires clinical validation and collaboration with healthcare institutions, the approach offers a promising foundation for improving data accessibility, interoperability, and the user experience of digital tools in Danish general practice.
\end{abstract}