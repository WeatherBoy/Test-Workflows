\section{Introduction}
This project is, in large part, a continuation of my Master's thesis ~\cite{thesis}, which concerned the development of digital applications for the healthcare sector. Throughout the process of writing my thesis, I became increasingly aware that the bottlenecks wouldn't be in developing the software. Instead, the main impediment was twofold; one was bureaucratic in nature - a human factor - while the second was technical in nature - the accessibility, or perhaps more so the inaccessibility, of data.
\\
The bureaucratic aspect consisted of miscommunication concerning application functionality between myself and stakeholders from the healthcare sector, in addition to misalignment between individual goals amongst personnel and broader goals at the administrative level from the industry side. Yet, as I have touched on this subject plentily in my thesis, I will refrain from diving further into it in this project. 
\\
\\
On the other hand, the technical difficulties, namely, data inaccessibility, are no less relevant. Therefore, following my thesis, I have once more reached out to the industry - the healthcare sector - in an attempt to assist in bridging the gap between developers and healthcare data. Seeing as data inaccessibility is not only an interfacing issue, but, maybe more prevalently, also a concern of standardisation, expert feedback, or collaboration, would contribute enormously to the endeavour of data accessibility.
\\
However, it proved extremely difficult to establish an industry contact. Therefore, this project primarily concerns itself with user experience (UX) before, during, and after a patient-medic checkup. Here, I have analysed the UX aspects of data broadcast through HTML reports and whether the user found the report adequate for a clinical setting. Furthermore, I have explored an American data standard, dubbed the Fast Healthcare Interoperability Resources (FHIR), and how the data used in the aforementioned HTML reports could be retrieved using FHIR.
Due to the nature of the project, the user in question, for these analyses, has been the healthcare personnel. And as I was unable to collaborate with the industry, I have instead been very fortunate in getting to interview and cooperate with a handful of my friends who are all studying to become medical practitioners. 
\\
\\
As such, this project aims to answer the questions:
\begin{itemize}
    \item Is it possible to define a common care pathway for multiple patients that broadly requires the same data for their medical practitioners?
    \item Can such data be presented in a standardized HTML report that is adequate for clinical use?
    \item Can the data required for the HTML report be retrieved using FHIR?
\end{itemize}   
