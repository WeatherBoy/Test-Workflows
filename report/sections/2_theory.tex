\section{Theory}\label{sec:theory}
\subsection{Fast Healthcare Interoperability Resources (FHIR)}
Fast Healthcare Interoperability Resources (FHIR) is a standard for healthcare data, designed with the intent of enabling secure digital exchange.
\\
Unlike most other products and industries, \emph{standards} thrive on monopoly. Monoculturalism is a widely debated topic and probably most commonly agreed upon as unhealthy or counterproductive in the grand scheme of things. In that sense, we, as humans and a collective, benefit from the development of multiple languages and cultures. However, when it comes to standards, especially something as critical as healthcare data, we all benefit from speaking exactly one language\footnote{One might be tempted to consult the XKCD comic of appendix \autoref{sec:XKCD-standards}.}. A singular language, one standard, enables seamless transfer of information between different institutions and even across country borders. This, in turn, permits third-party developers to more readily design software applications harnessing the statistical advantages associated with large quantities of standardised data. Furthermore, if everybody utilised one standard, these technological advancements could also propagate across institutions and national borders. Taking into account this strange tolerance of monopolies, in order to understand why FHIR is \textbf{the} standard, one must necessarily also get acquainted with the organisation developing it, Health Level Seven (HL7).
\\
FHIR was created and is currently maintained by the American non-profit organisation, HL7. It was founded in $1987$ by Donald W. Simborg with the original objective of developing a standard for the exchange of data within Hospital Information Systems\footnote{HIS is a somewhat broad term, describing the elements of health informatics which mainly focus on the administrative needs of hospitals. A HIS is a comprehensive integrated system, which has the intended function of managing all operational aspects of a hospital, i.e. medical, administrative, financial, legal and issues pertaining to the processing of services. HIS are also commonly referred to as Hospital Management Software or a Hospital Management System~\cite{Wiki-HIS}.} (HIS)~\cite{Wiki-HL7}. As of $2025$, today HL7's vision, as stated on their official website~\cite{HL7-about}, is:
\begin{center}
    \emph{"A world where every individual has secure, timely, and appropriate access to accurate health information, empowering them, their care teams, and communities to make informed decisions anytime, anywhere."}
\end{center}
As a testament to their vision, in September $2012$ HL7 announced their intention to license their standards at no cost, and finally in April $2013$ they made all their primary standards, alongside other select products, of which FHIR is a part, licensable at no cost~\cite{HL7-no-cost}. Furthermore, as an American-founded and American-based organisation, HL7 is accredited by the American National Standards Body (NSB) of the International Organisation for Standardisation (ISO), namely the American National Standards Institute (ANSI). 	