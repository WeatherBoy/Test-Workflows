\section{Methodology}
\subsection{Reaching out to the industry}
While writing my thesis, I hoped that the software solutions I was developing could build the foundation for a start-up. However, multiple complications arose. I ended up working on three different projects, and even the "easiest", at least from a software development point of view, quickly hit a wall. It ultimately proved too difficult to implement, simply because the data sources were too inaccessible. While the two other projects might have worked in theory, they would still be riddled with infant diseases as soon as they reached production, due to the data inaccessibility issues. Seeing as the healthcare sector is already full of disjoint software solutions, usually designed with clunky or, at best, simply odd workarounds, I elected to refrain from contributing my own disjoint software solution and thus abandoned the start-up dream.
\\
Yet, it remained clear to me that this underlying issue of data inaccessibility persisted. Therefore, I became quite hopeful upon reading the 2024 report by "Danske Regioner" ~\cite{Den-Reg-digitalisation}. In the report, "Danske Regioner" proposed a so-called "Danish Healthcare Cloud". I have since come to think of the \textbf{Healthcare Cloud} as a pseudo-appstore; however, it might be more aptly described as a combination of a data warehouse, or data lake, and an appstore. The core functionality of the \textbf{Healthcare Cloud} would be a platform which allowed for easy data sharing. In other words, all applications on the \textbf{Healthcare Cloud} would be required to store their data in the \textbf{Healthcare Cloud}, such that any other application on the \textbf{Healthcare Cloud} would have the same data readily accessible. This would immediately eliminate any need for the healthcare personnel to perform time-wasting double (triple or even more) loggings of the data. However, the \textbf{Healthcare Cloud} would also need to support a universal login for the healthcare personnel, such that they don't have to keep separate log-ins for each platform. Additionally, the \textbf{Healthcare Cloud} would have to handle some kind of licensing to verify that it is the right developers who can create platforms with access to the data.
\\
Upon finishing my thesis in March of 2025, I sought to initiate contact with "Danske Regioner" in the hopes that I might be able to contribute to the project. Unfortunately, "Danske Regioner" isn't currently working on that project. Instead, they had simply collaborated with a tech consultancy to outline the \textbf{Healthcare Cloud}, what problems it could solve and how it could be developed. "Danske Regioner" responded kindly to my message and advised that I should reach out to CIMT ("Center for IT \& Medico-Teknologi") in the capital region. They deemed that CIMT might be able to allocate resources for such a project.
\\
It took roughly two months before I got to have a meeting with CIMT. They were two very nice and formal representatives who, in some capacity, worked with planning and scheduling data within the healthcare sector of the capital region. However, although they also found the project of interest, they very quickly shut it down. They informed that CIMT had neither the resources nor the authority to take on a project such as the \textbf{Healthcare Cloud}.
\\
Another big aspect of realising the \textbf{Healthcare Cloud} is data standardisation. Unless the healthcare institutions can somehow come to an agreement on how to label all data points, the \textbf{Healthcare Cloud} would still not allow for cross-institutional data sharing. So even if CIMT managed to standardise the healthcare data within the capital region, it would remain incompatible with the data from the other four regions\footnote{The five regions being: The Capital Region, The Region of Northern Jutland, The Region of Central Jutland, The Region of Southern Denmark, and The Region of Zealand}. As such, CIMT ultimately rerouted me back to "Danske Regioner", as, if anybody had to "make the decree", so to speak, of data standardisation, it would be them.
\\
Upon reaching out to "Danske Regioner" once more, I was, to my great joy, met by a very kind secretary, who remembered my original call. She seemed impressed that I had managed to secure a meeting with CIMT, and she promptly directed me further up the administrative chain. Luckily, this time it was simply a phone call, and it didn't take two months to plan. However, the administrative worker I got a hold of at "Danske Regioner" stated that, although it was the staff at CIMT's impression, "Danske Regioner" doesn't handle administrative software on that kind of scale. Instead, I was redirected to the Danish Health Data Authority (DHDA).
\\
I was fully aware that DHDA wasn't working on a \textbf{Healthcare Cloud}, so when I reached out to them, it was more so in the hopes of creating a short collaboration for a 10 ECTS course at DTU. Maybe they could provide some actual or mock data from the Danish healthcare sector, and hopefully, I could contribute something useful with that. Unfortunately, that wasn't the case. DHDA found my motivation endearing and recommended that I apply for a job at their department, as they hoped there would come an opening where I would have the possibility of working on something adjacent to the \textbf{Healthcare Cloud}. Nevertheless, DHDA proved a dead end all the same, and as such, I was unable to secure data for this project from an official source. 

\subsection{Proceeding without clinical data}

