\section{Conclusion}
In this report I have addressed the issue concerning data standardisation and accessibility within the Danish healthcare sector. In particular, in an attempt to alleviate some of the everyday tasks for healthcare personnel, I set out to answer the following questions:
\begin{itemize}
    \item Is it possible to define a common care pathway for multiple patients that broadly requires the same data for their GPs?
    \item Can such data be presented in a standardised HTML report that is adequate for clinical use?
    \item Can the data required for the HTML report be retrieved using FHIR?
\end{itemize} 
To this end, I have researched the FHIR standard and its applicability for retrieving patient data, in addition to designing and implementing a prototype HTML report for clinical use and conducting UX interviews with medical students.
\\
Regarding the first question, although I didn't find a common care pathway as I had initially expected, I did find that both diabetes patients and pregnant women serve as good examples of patient groups that broadly require the same data for their GPs. In particular, I tackled the diabetes use case, and hope that the findings can provide a foundation for further work on standardised reports for other patient groups.
\\
With respect to the second question, I wrote a codebase to automatically generate an HTML report based on questionnaire data in a JSON format. Although I believe this to be a proof of concept for a standardised HTML report, I lacked the necessary clinical resources to properly validate the report for clinical use. However, through UX interviews with medical students, I was able to gather valuable feedback and insights that can be used for further development of the concept.
\\
Finally, regarding the third question, I researched the FHIR standard and found that it is indeed possible to retrieve the necessary data for the HTML report using FHIR resources as exemplified by the datamodel diagram in \autoref{fig:datamodel-diagram} of the analysis section.
